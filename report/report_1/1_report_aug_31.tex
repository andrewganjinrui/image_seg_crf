\documentclass[a4paper]{article}

\usepackage[a4paper,bindingoffset=0.2in,%
left=1in,right=1in,top=1in,bottom=1in,%
footskip=.25in]{geometry}

%opening
\title{Image Segmentation using Conditional Random Fields \\
	  $[\textrm{CSE 471}]$ Statistical Methods in AI}
\author{Santosh K (201150883), Sudheer T (201550851), Srikanth M (20163014)}
\date{}

\begin{document}

\maketitle

\begin{abstract}
Given a image containing object(s), the task is to segment the image based on
objects and label each segmented object. In this project, we plan to replicate 
the work done in \cite{icml_2009}, where in the authors perform multi-class 
image segmentation using conditional random fields. Further, they use `global 
features for image classification' as a complementary information along with the 
local features to obtain better segmentation. In the current project, our 
primary focus is on understanding and implementing conditional random fields 
(CRFs) for the task of object based image segmentation.
\end{abstract}


\section{Markov random fields}
\label{sec_mrf}

Markov random fields (MRFs) are generative models, which can be used to model 
the joint distribution of the data (features) $\mathbf{X}$ and the labels 
(multi-class) $\mathbf{Y}$, i.e., $P(\mathbf{X, Y})$. Once we obtain the joint 
distribution, we can easily obtain the conditional distributions $P(\mathbf{Y} 
\mid \mathbf{X})$ and $P(\mathbf{X} \mid \mathbf{Y})$, but it often difficult to 
model the joint distribution, especially when the random variables are 
continuous and dependent on each other.

MRFs can also be used in unsupervised scenario, just to model the distribution 
of features $\mathbf{X}$. It is important to note that the features (color and, 
texture based) are highly correlated (dependent on) with each other, 
and hence we need random fields to model such data. The factorization of the 
joint distribution of such data (features) can expressed with the help of 
\textit{potential functions}, $\varphi_C(\mathbf{x}_c)$ \cite{Bishop:2006:ML}.
\begin{equation}
 p(\mathbf{x}) = \frac{1}{Z} \prod_C \varphi(\mathbf{x}_C),
\end{equation}
where each potential function $\varphi(\mathbf{x}_C)$ represents a maximal 
clique in the graph defined by the random field, and $Z$ is called as partition 
function which acts as a normalization constant.


\section{Conditional random fields}
\label{sec_crf}
Conditional random fields model only the probability distribution over target
labels (in our case) given the data i.e., $P(\mathbf{Y} \mid \mathbf{X})$. This 
is primarily useful, when we have labeled data and our task is only prediction, 
i.e., given a test feature vector $\mathbf{X}_t$, we would like to obtain the 
distribution (posterior probability) over the target labels, $P(\mathbf{Y} \mid 
\mathbf{X}_t)$.

CRFs do not model the data (the factorization of the data does not matter), and 
so they are less complex than MRFs. This also makes CRFs limited on to the
prediction task.

\section{Approach}
\label{sec_approach}

We follow the approach as described in \cite{icml_2009} and the steps are 
briefly described as follows:

\begin{enumerate}

\item Initially, apply an unsupervised segmentation technique to partition 
the complete image into number of patches on multiple scale levels. Each scale 
level gives different granularity in terms of color, and each patch represents
a set of neighboring pixels. Initially we use simple $k$-means to as the 
clustering technique.

\item For each patch on each level, obtain color, texture and SIFT (scale 
invariant feature transform) features. These features are dependent on each 
other.

\item From all patch features, compute a global feature vector.

\item Train SVMs to predict the class label for each path and also for the 
complete image.

\item Build a CRF to model the dependencies between obtained patch labels, 
given the feature vectors.

\item Applying a threshold on the posteriors of the labels gives the final 
segmentation.

\end{enumerate}


\section{Project plan and milestones}

\begin{enumerate}
  \item Sep 24, 2016: Implementation the pipeline as described in Section 
\ref{sec_approach}, with primary focus on CRF and using existing libraries for 
feature extraction and SVM.

  \item Oct 31, 2016: Depending on the progress and obtained results, we also 
plan to implement the MRFs for unsupervised image segmentation.

\end{enumerate}

\bibliographystyle{plain}
\bibliography{refs}
\end{document}
